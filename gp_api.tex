\documentclass[12pt]{article}
\usepackage{style/fullpage}
\usepackage{style/underscore}
\usepackage{style/environ}
\usepackage{color}
\usepackage{alltt}
\usepackage{graphicx}
\usepackage{etoolbox} % for @for \do ??
\usepackage[hidelinks]{hyperref} % for hyperref
\usepackage{xstring}
\usepackage{adjustbox}
\usepackage{etoolbox}

\newbool{CapiInclude}
\newbool{FapiInclude}
\newbool{CxxapiInclude}
\newbool{PyapiInclude}

\input{preamble/settings}

%\IfFileExists{minted.sty}{
%\usepackage{minted}
%\newminted[ccode]{c}{}
%\newminted[cxxcode]{c++}{}
%\newminted[fcode]{fortran}{}
%\newminted[pycode]{python}{}
%\newmint[cfuncsig]{c}{}
%\newmint[cxxfuncsig]{c++}{}
%\newmint[ffuncsig]{fortran}{}
%\newmint[pyfuncsig]{python}{}
%}{
\newcommand{\commentOut}[1]{}
\usepackage{listings}
\lstnewenvironment{ccode}
    {\lstset{language=C, breaklines=true, breakautoindent=true}}
    {}
\lstnewenvironment{cxxcode}
    {\lstset{language=C++, breaklines=true, breakautoindent=true}}
    {}
\lstnewenvironment{fcode}
    {\lstset{language=Fortran, breaklines=true, breakautoindent=true}}
    {}
\lstnewenvironment{pycode}
    {\lstset{language=Python, breaklines=true, breakautoindent=true}}
    {}
%}

\newcommand{\apih}[2]{
\noindent
\section*{#1}
\label{api:#1}
\textit{#2}
}

%\newcommand{\optype}[1]{
%\newline
%\textbf{Operation Type:}#1
%\newline
%}

\newcommand{\wcoll}{
\noindent
\textcolor{blue}{Operation Type:} Collective on the world processor group
\newline
}
\newcommand{\dcoll}{
\noindent
\textcolor{blue}{Operation Type:} Collective on the default processor group
\newline
}
\newcommand{\gcoll}{
\noindent
\textcolor{blue}{Operation Type:} Collective on the processor group inferred from the arguments
\newline
}
\newcommand{\ncoll}{
\noindent
\textcolor{blue}{Operation Type:} One-sided (non-collective)
\newline
}
\newcommand{\local}{
\noindent
\textcolor{blue}{Operation Type:} Local
\newline
}

\newenvironment{desc}{
\noindent
\textcolor{blue}{API description:}\par}{}

\newenvironment{cdesc}{
\noindent
\textcolor{blue}{C API description:}\par}{}

\newenvironment{fdesc}{
\noindent
\textcolor{blue}{Fortran API description:}\par}{}

\newenvironment{cxxdesc}{
\noindent
\textcolor{blue}{C++ API description:}\par}{}


\ifbool{FapiInclude}{
\newenvironment{fapi}{
\noindent
\textcolor{blue}{Fortran Interface}
\small
}{}
}{\NewEnviron{fapi}{}}

\ifbool{CapiInclude}{
\newenvironment{f2dapi}{
\noindent
\textcolor{blue}{Fortran 2D Interface}
\small
}{}
}{\NewEnviron{f2dapi}{}}

\newenvironment{pydesc}{
\noindent
\textcolor{blue}{Python API description:}

}{}

\ifbool{CapiInclude}{
\newenvironment{capi}{
\noindent
\textcolor{blue}{C Interface}
\small
}{}
}{\NewEnviron{capi}{}}

\ifbool{CxxapiInclude}{
\newenvironment{cxxapi}{
\noindent
\textcolor{blue}{C++ Interface}
\small
}{}
}{\NewEnviron{cxxapi}{}}

\ifbool{PyapiInclude}{
\newenvironment{pyapi}{
\noindent
\textcolor{blue}{Python Interface}
\small
}{}
}{\NewEnviron{pyapi}{}}

\newenvironment{funcargs}{\noindent\begin{tabular}{@{}p{0.16\textwidth}@{}p{0.14\textwidth}@{}p{0.58\textwidth}@{}p{0.08\textwidth}@{}}}{\end{tabular}}

\newcommand{\funcarg}[4]{\texttt{#1}&\texttt{#2}&\texttt{#3}&\texttt{#4}\\}

\newcommand{\inarg}[3]{\texttt{#1}&\texttt{#2}&\texttt{#3}&\textcolor{blue}{\texttt{input}}\\}

\newcommand{\inoutarg}[3]{\texttt{#1}&\texttt{#2}&\texttt{#3}&\texttt{\textcolor{blue}{in}/\textcolor{red}{out}}\\}

\newcommand{\outarg}[3]{\texttt{#1}&\texttt{#2}&\texttt{#3}&\textcolor{red}{\texttt{output}}\\}

\definecolor{copper}{RGB}{213,117,0}
\definecolor{silver}{RGB}{112,114,118}
\definecolor{bronze}{RGB}{168,60,15}
\definecolor{gold}{RGB}{241,171,0}
\definecolor{onyx}{RGB}{36,36,36}
\definecolor{platinum}{RGB}{178,179,181}

\newcounter{SeeAlsoCounter}
\makeatletter
\newcommand*{\seealso}[1]{%
\noindent
\textcolor{blue}{See Also:}
\newline
\def\mystring{#1}
\StrCount{\mystring}{,}[\commacount]
\setcounter{SeeAlsoCounter}{0}
\@for\arg:=#1\do{%
  \hyperref[api:\arg]{\nameref*{api:\arg}}\ifnum\value{SeeAlsoCounter}<\commacount,\fi
  \stepcounter{SeeAlsoCounter}
}%
}
\makeatother

\graphicspath{{./figures/}}

\begin{document}


\apih{ACCESS ELEMENT}{Obtain a pointer to a locally held GP element}

\begin{capi}
\begin{ccode}
void GP_Access_element(int g_p, int *subscript, void *ptr, int *size)
\end{ccode}
\begin{funcargs}
\inarg{int}{g_p}{pointer array handle}
\inarg{int*}{subscript}{indices of array element}
\outarg{void*}{ptr}{pointer to locally held element}
\outarg{int*}{size}{size in bytes of locally held element}
\end{funcargs}
\end{capi}

\local

\begin{desc}

This function returns a pointer to the data element element represented by the
array subscript. The suscript must be local to the processor, which means that
it must be within the bounds returned by the GP_Distribution function for the
calling process. It is an error if the subscript values represent an element
on a different processor.

\end{desc}

\seealso{DISTRIBUTION, RELEASE ELEMENT, RELEASE UPDATE ELEMENT}

\apih{ALLOCATE}{Allocate resources for a new GP array}

\begin{capi}
\begin{ccode}
int GP_Allocate(int g_p)
\begin{funcargs}
\inarg{int}{g_p}{pointer array handle}
\end{funcargs}
\end{ccode}
\end{capi}

\local

\begin{desc}

This function is called after a new array handle has been created and its
attributes have been set. This function actually allocates resources for the GP
array and allows it to be used by the program to store data.

\end{desc}

\seealso{CREATE HANDLE, SET DIMENSIONS}

\apih{ASSIGN LOCAL ELEMENT}{Assign a pointer to local data}

\begin{capi}
\begin{ccode}
void GP_Assign_local_element(int g_p, int *subscript, void *ptr, int size)
\end{ccode}
\begin{funcargs}
\inarg{int}{g_p}{pointer array handle}
\inarg{int*}{subscript}{indices of array element}
\inarg{void*}{ptr}{pointer to local data}
\inarg{int}{size}{size in bytes of local data}
\end{funcargs}
\end{capi}

\local

\begin{desc}

This function is used to associate a GP array element located locally on the
process with a memory segment. The memory must be allocated using the GP_Malloc
function and the size argument used in this call must correspond to the size
argument in the GP_Malloc call. It is an error if the subscript indices fall
outside the range of values returned by the GP_Distribution call for the calling
process. 

\end{desc}

\seealso{DISTRIBUTION, MALLOC}

\apih{CREATE HANDLE}{Create a GP handle}

\begin{capi}
\begin{ccode}
int GP_Create_handle()
\end{ccode}
\end{capi}

\local

\begin{desc}

This function creates a new GP handle and returns it to the user. Calling this
function is the first step in creating a new GP array.

\end{desc}

\seealso{ALLOCATE}

\apih{DESTROY}{Remove a GP array from the system and free up resources}

\begin{capi}
\begin{ccode}
int GP_Destroy(int g_p)
\begin{funcargs}
\inarg{int}{g_p}{pointer array handle}
\end{funcargs}
\end{ccode}
\end{capi}

\local

\begin{desc}

Destroy a GP array and free up the associated resources. Not that this function
does not remove individual array elements. These should be removed first by a
call to GP_Free_local_element and GP_Free. The first call returns the pointer to
a local element and sets the corresponding pointer and size in the GP to zero
and the second actually removes the local memory segment. The complete code for
properly destroying a hypothetical 2-dimensional GP array is

\begin{verbatim}
   GP_Distribution(g_p, me, lo, hi);
   for (i = lo[0]; i<=hi[0]; i++) {
     subscript[0] = i;
     for (j = lo[1]; j<=hi[1]; j++) {
       subscript[1] = j;
       GP_Free(GP_Free_local_element(g_p, subscript);
     }
   }
   GP_Destroy(g_p);
\end{verbatim}

\end{desc}

\seealso{FREE, FREE LOCAL ELEMENT}

\apih{DISTRIBUTION}{Report array bounds for elements held locally on specified
processor}

\begin{capi}
\begin{ccode}
void GP_Distribution(int g_p, int proc, int *lo, int *hi)
\begin{funcargs}
\inarg{int}{g_p}{pointer array handle}
\inarg{int}{proc}{requested processor}
\outarg{int*}{lo}{lower indices of locally held data}
\outarg{int*}{hi}{upper indices of locally held data}
\end{funcargs}
\end{ccode}
\end{capi}

\local

\begin{desc}

Return the bounding indices of the data held locally on processor proc. For the
calling processor, this data can usually be accessed much more quickly than data
on remote processors.

\end{desc}

\apih{FREE}{Free up memory allocated using GP_Malloc}

\begin{capi}
\begin{ccode}
void GP_Free(void *ptr)
\begin{funcargs}
\inarg{void*}{ptr}{pointer to local memory segment}
\end{funcargs}
\end{ccode}
\end{capi}

\local

\begin{desc}

This function is used to free up local memory segments allocated using
GP_Malloc. Note that it should not be using on segments allocated using ordinary
malloc.

\end{desc}

\seealso{MALLOC, DESTROY, FREE LOCAL ELEMENT}

\apih{FREE LOCAL ELEMENT}{Remove association between local memory segment and GP array}

\begin{capi}
\begin{ccode}
void* GP_Free_local_element(int g_p, int *subscript)
\begin{funcargs}
\inarg{int}{g_p}{pointer array handle}
\inarg{int*}{subscript}{subscript to local array element}
\end{funcargs}
\end{ccode}
\end{capi}

\local

\begin{desc}

This function removes the association between a local memory segment and the
corresponding local GP array element and returns a pointer to the local memory
segment. Future requests for that array element
will return a null pointer and zero size. Note that this does not remove the
memory segment itself, which must be done by a separate call to GP_Free.

\end{desc}

\seealso{FREE, DESTROY}

\end{document}
