\usepackage{etoolbox} % for @for \do
\usepackage[hidelinks]{hyperref} % for hyperref

%\IfFileExists{minted.sty}{
%\usepackage{minted}
%\newminted[ccode]{c}{}
%\newminted[cxxcode]{c++}{}
%\newminted[fcode]{fortran}{}
%\newminted[pycode]{python}{}
%\newmint[cfuncsig]{c}{}
%\newmint[cxxfuncsig]{c++}{}
%\newmint[ffuncsig]{fortran}{}
%\newmint[pyfuncsig]{python}{}
%}{
\usepackage{listings}
\lstnewenvironment{ccode}
    {\lstset{language=C, breaklines=true, breakautoindent=true}}
    {}
\lstnewenvironment{cxxcode}
    {\lstset{language=C++, breaklines=true, breakautoindent=true}}
    {}
\lstnewenvironment{fcode}
    {\lstset{language=Fortran, breaklines=true, breakautoindent=true}}
    {}
\lstnewenvironment{pycode}
    {\lstset{language=Python, breaklines=true, breakautoindent=true}}
    {}
%}

\newcommand{\apih}[2]{
%\hspace*{\fill} \\ \\ \\
%\color{black}
\noindent
\section*{#1}
\label{api:#1}
\textit{#2}
%\textbf{\Large{#1} -- \normalsize{#2}}
\newline
}

%\newcommand{\optype}[1]{
%\newline
%\textbf{Operation Type:}#1
%\newline
%}

\newcommand{\wcoll}{
\noindent
\textcolor{blue}{Operation Type:} Collective on the world processor group
\newline
}
\newcommand{\dcoll}{
\noindent
\textcolor{blue}{Operation Type:} Collective on the default processor group
\newline
}
\newcommand{\gcoll}{
\noindent
\textcolor{blue}{Operation Type:} Collective on the processor group inferred from the arguments
\newline
}
\newcommand{\ncoll}{
\noindent
\textcolor{blue}{Operation Type:} One-sided (non-collective)
\newline
}
\newcommand{\local}{
\noindent
\textcolor{blue}{Operation Type:} Local
\newline
}

\newenvironment{desc}{
\noindent
\textcolor{blue}{API description:}

}{}

\newenvironment{cdesc}{
\noindent
\textcolor{blue}{C API description:}

}{}

\newenvironment{fdesc}{
\noindent
\textcolor{blue}{Fortran API description:}

}{}

\newenvironment{cxxdesc}{
\noindent
\textcolor{blue}{C++ API description:}

}{}

\newenvironment{fapi}{
\noindent
\textcolor{blue}{Fortran Interface}
\small
}{
}

\newenvironment{f2dapi}{
\noindent
\textcolor{blue}{Fortran 2D Interface}
\small
}{
}

\newenvironment{capi}{
\noindent
\textcolor{blue}{C Interface}
\small
}{
}

\newenvironment{cxxapi}{
\noindent
\textcolor{blue}{C++ Interface}
\small
}{
}

\newenvironment{pyapi}{
\noindent
\textcolor{blue}{Python Interface}
\small
}{
}

\newenvironment{funcargs}{
\begin{tabular}{ p{0.15\textwidth} p{0.15\textwidth} p{0.5\textwidth} p{0.1\textwidth} }
}{
\end{tabular}
}

\newcommand{\funcarg}[4]{
\texttt{#1} & \texttt{#2} & \texttt{#3} & \texttt{#4} \\
}

\definecolor{copper}{RGB}{213,117,0}
\definecolor{silver}{RGB}{112,114,118}
\definecolor{bronze}{RGB}{168,60,15}
\definecolor{gold}{RGB}{241,171,0}
\definecolor{onyx}{RGB}{36,36,36}
\definecolor{platinum}{RGB}{178,179,181}

\makeatletter
\newcommand*{\seealso}[1]{%
\noindent
\textcolor{blue}{See Also:}
  \@for\arg:=#1\do{%
    \hyperref[api:\arg]{\nameref*{api:\arg}}
  }%
}
\makeatother

