\documentclass[12pt]{article}
\usepackage{fullpage}
\usepackage{color}
\usepackage{alltt}
\usepackage{underscore}
\usepackage{environ}
\usepackage{graphicx}
\newcommand{\apih}[1]{
\hspace*{\fill} \\ \\ \\
\color{black}
\noindent
\textbf{\LARGE
#1
\newline
}
}

\newcommand{\access}[1]{\textcolor{green}{#1}}


\newenvironment{fapi}{
\noindent
\textcolor{blue}{Fortran Interface}
\begin{alltt} 
\small
\color{red}
}{
\end{alltt}
}

\newenvironment{f2dapi}{
\noindent
\textcolor{blue}{Fortran 2D Interface}
\begin{alltt} 
\small
\color{red}
}{
\end{alltt}
}

\newenvironment{capi}{
\noindent
\textcolor{blue}{C Interface}
\begin{alltt} 
\small
\color{red}
}{
\end{alltt}
\color{black}
}

\newenvironment{cxxapi}{
\noindent
\textcolor{blue}{C++ Interface}
%\vspace{-18pt}
\begin{alltt} 
\small
\color{red}
}{
\end{alltt}
}




\begin{document}

\section{Additional Explanations}

Global arrays are distributed array objects supported in a message-passing program through the GA library calls. They can be created, destroyed, and manipulated using a set of GA operations.

\subsection{Attributes of GA Operations}

Global arrays can have the following attributes.

\begin{itemize}
\item One-sided/independent - references shared or remote data without remote process cooperation (unlike send/receive pair)
\item Collective - requires all processes to make the call
\item Local - operation is local to each process and does not require communication
\item Atomic - operation has mutual exclusion built in 
\item Non-blocking - Nonblocking operations initiate a communication call and then return control to the application. A return from a nonblocking operation call indicates a mere initiation of the data transfer process and the operation can be completed locally by making a call to the wait (e.g., nga_nbwait) routine.
\end{itemize}
 
\subsection{Language Interoperability}

GA provides C and Fortran interfaces  to the same array objects in the same (mixed-language) program.

\subsection{Higher-dimensional Array API}

All releases of GA less than 3.0 supported only 2-dimensional arrays. Starting with the release of 3.0, GA arrays can be be essentially of arbitrary dimensions but the package as compiled supports up to 7 dimensions (Fortran limit for multidimensional arrays).

The 2-D API for backward compatibility with older versions of the toolkit remained unchanged. For some GA operations, two versions of the API are available. They are identified with a ``GA" or ``NGA" prefix that correspond to either the old 2-dimensional or the new arbitrary(N)-dimensional version, respectively. 

\end{document}